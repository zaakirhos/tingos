
\documentclass[12pt]{article}
 
\usepackage{enumitem}
\usepackage[margin=1in]{geometry} 
\usepackage{amsmath,amsthm,amssymb,scrextend}
\usepackage{fancyhdr}
\usepackage{alltt}
\pagestyle{fancy}

 


 

\begin{document}


\lhead{CS 35}
\chead{Zakir Hossain, Summit Pradhan}
\rhead{\today}
 
% \maketitle
 
\section{Inductive Proofs}

\text{Solve the following problems with Induction}\\
 
\begin{enumerate}

\item 
\text{The sum of the first n even numbers is } $n^2 + n$. 
\text{That is}, $$\sum_{i=1}^{n} 2i =  n^2 + n$$

\textit{proof.}

\textbf{Base case (n = 1):  } 

$$\text{LHS:  }\sum_{i=1}^{1}{2i} = 2$$
$$\text{ RHS:  }1^2 +1 = 2 $$ \\

\text{Since LHS = RHS, the base case holds true}

\textbf{Inductive Hypothesis: } Assume that

$$\sum_{i=1}^{n} 2i =  n^2 + n \text{   for all n such that } 1\le n \le k$$

\textbf{Inductive Step: } 
$$\text{I must show that  }\sum_{i=1}^{k+1} 2i =  (k+1)^2 + (k+1) = k^2 + 3k + 2$$


\text{If we expand the last term, we get: }
$$\sum_{i=1}^{k+1} 2i = 2(k+1) + \sum_{i=1}^{k} 2i$$
\text{ With our inductive hypothesis, we have: }
$$ = 2(k+1) +k^2 + k $$
$$= k^2 + k + 2k + 1$$
$$= k^2 + 3k + 2 $$\\


\item 
$$\sum_{i=1}^{n} \frac{1}{2^i} =  1 - \frac{1}{2^n}$$

\textit{proof.}

\textbf{Base case (n = 1):  } 

$$\text{LHS:  }\sum_{i=1}^{1} \frac{1}{2^i} = \frac{1}{2}$$
$$\text{ RHS:  } 1 - \frac{1}{2^1} = \frac{1}{2} $$ \\

\text{Since LHS = RHS, the base case holds true}

\textbf{Inductive Hypothesis: } Assume that

$$\sum_{i=1}^{n} \frac{1}{2^i} =  1 - \frac{1}{2^n} \text{   for all n such that } 1\le n \le k$$

\textbf{Inductive Step: } 
$$\text{I must show that  } \sum_{i=1}^{k+1} \frac{1}{2^i} = 1 - \frac{1}{2^{k+1}}$$

\text{If we expand the last term, we get: }
$$\sum_{i=1}^{k+1} \frac{1}{2^i} = \frac{1}{2^{k+1}}  + \sum_{i=1}^{k} \frac{1}{2^i}$$
\text{ With our inductive hypothesis, we have: }
$$ = \frac{1}{2^{k+1}} + 1 - \frac{1}{2^k} $$
\text{Finding common denominator (with expo law $2^k * 2^m = 2^{k+m}$ ): }
$$= \frac{1}{2^{k+1}} + 1 - \frac{1}{2^k} *  \left (\frac{2^1}{2^1}\right)$$

$$= \frac{1}{2^{k+1}} + 1 - \frac{2}{2^{k+1}}$$

$$=  1 - \frac{1}{2^{k+1}}$$


\pagebreak
\item 
$$\sum_{i=0}^{n} 2^i =  2^{n + 1} - 1$$

\textit{proof.}

\textbf{Base case (n = 0):  } 

$$\text{LHS:  }\sum_{i=0}^{0}{2^i} = 1$$
$$\text{ RHS:  } 2^{0+1} - 1 = 2 -1 = 1 $$ \\

\text{Since LHS = RHS, the base case holds true}

\textbf{Inductive Hypothesis: } Assume that

$$\sum_{i=0}^{n} 2^i =  2^{n + 1} - 1 \text{   for all n such that } 1\le n \le k$$

\textbf{Inductive Step: } 
$$\text{I must show that  }\sum_{i=0}^{k+1} 2^i =  2^{(k+1) + 1} - 1  = 2^{k+2} - 1$$


\text{If we expand the last term, we get: }
$$\sum_{i=0}^{k+1} 2^i = 2^{k+1} + \sum_{i=0}^{k} 2^i $$
\text{ With our inductive hypothesis, we have: }
$$ = 2^{k+ 1} + 2^{k + 1} - 1$$


$$= k^2 + k + 2k + 1$$
$$= 2 * (2^{k+1}) -1 $$
\text{2 is same as $2^1$ (to apply law $2^k * 2^m = 2^{k+m})$}
$$= 2^1 * (2^{k+1}) -1 $$
$$= 2^{(k+1+1) } -1  = 2^{k+2 } -1$$

\end{enumerate}
\pagebreak
\vspace{1cm}
\section{Recursive Invariants}
The function \texttt{minEven}, given below in pseudocode, takes as input an array $A$ of size $n$ of numbers.  It returns the smallest \textit{even} number in the array.  If no even numbers appear in the array, it returns positive infinity ($+\infty$).  Using induction, prove that the \texttt{minEven} function works correctly.  Clearly state your recursive invariant at the beginning of your proof.

\begin{alltt}
Function minEven(A,n)
  If n = 0 Then
    Return +\ensuremath{\infty}
  Else
    Set best To minEven(A,n-1)
    If A[n-1] < best And A[n-1] is even Then
      Set best To A[n-1]
    EndIf
    Return best
  EndIf
EndFunction
\end{alltt}


\textbf{Recursive Invariant:}

Let $P(n)$ be the function that represents minEven such that $P(n)$ either returns the lowest even number in the first n elements of an array or returns positive infinity
if no even numbers are within the first n elements. 

\bigskip
\textbf{Base Case (n = 0)}
\begin{enumerate}
    \item For $P(0)$, the function has an input of the first 0 elements in an array, or in other words no elements at all. 
    Since an empty array can never contain an even number, it must be true that $P(0)$ outputs positive infinity. 
    \item The first if statement in minEven would check if n is equal to zero. Since n is equal to zero, it returns positive infinity. 
    \item Therefore, the base case of n = 0 follows the recursive invariant.
\end{enumerate}

\bigskip
\textbf{Induction Hypothesis}

Assume that $P(n)$ functions properly for all inputs of size $0 <= n <= k$ such that $P(n)$ either returns the lowest even number in the 
first n terms of an array or that it returns positive infinity if the array only contains odd numbers in the first n elements. 

\bigskip
\textbf{Inductive Step}

Goal: Show that P(K+1) outputs the proper value
\begin{enumerate}
    \item The initial call of the function will be minEven(A, K+1) where A is an array of at least K+1 elements. 
    \item Since K+1 is greater than zero, the first if statement in minEven is false, causing the function to enter the else statement
    \item In the else statement, minEven(A, K+1) calls minEven(A, K) and sets it to the variable best.
    \begin{enumerate}
        \item  Based on the Inductive Hypothesis, it can be assumed that minEven(A, K) works properly and either returns
         the lowest even number in the first K elements or returns positive infinity. 
    \end{enumerate}
    \item In the second if statement, if A[K] is a smaller even number than best, then best is set to A[K]. 
    \begin{enumerate}
        \item Based on the previous step, it can be assumed that best is either the smallest even number from the first K elements or positive infinity if the 
        array does not contain any even numbers. If A[K] is a smaller even number than best, that means that A[K] must be the smallest even number in the first
         K+1 elements of the array. If this is true, then best is set to A[K]. If this is not true, then best retains its value. In both situations, the value stored 
         in best is either the lowest even number in the first K+1 elements of the array, or positive infinity if the array does not contain any even numbers. 
    \end{enumerate}
    \item  minEven(A,K+1) then returns best.
    \begin{enumerate}
        \item As stated in the previous step, best represents either the smallest even number in the first K+1 elements of the array or positive infinity if there were no even numbers in that range. 
        Since best is equal to the value that minEven is intended to output, it must be true that minEven(A, K+1) works as intended. 
    \end{enumerate}
\end{enumerate}

Since P(K+1) outputs the intended results, the inductive step is proven to be true. 
\end{document}
